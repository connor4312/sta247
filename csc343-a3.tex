\documentclass{article}
\usepackage{amsmath,amssymb,graphicx,tikz}
\usepackage{hyperref}
\usepackage{tikz}
\usepackage{amsfonts}
\usepackage{amsmath}

\title{CSC343 Fall 2015, Assignment 3}
\author{Connor Peet \#1001088208}
\renewcommand{\today}{~}
\hypersetup{pdfpagemode=Fullscreen,
  colorlinks=true,
  linkfileprefix={}}
\begin{document}
\maketitle

\begin{enumerate}
\item
    \begin{enumerate}
    \item [(a)]
        \begin{itemize}
        \item $BH \rightarrow AD$ is fine. $BH^+$ includes all columns.
        \item $D \rightarrow BH$ is fine. $D^+$ includes all columns.
        \item $BCE \rightarrow F$ is \textbf{invalid}; its closure only contains $CF$.
        \item $F \rightarrow C$ is \textbf{invalid}; its closure only contains $C$.
        \item $A \rightarrow GEF$ is \textbf{invalid}; its closure only contains $CEFG$.
        \end{itemize}
    \item [(b)]
        \begin{itemize}
        \item Decompose using $F \rightarrow C$. $F^+ = C$, so we have $R_1 = FC$ and $R_2 = ABDEFGH$. The FD projection on $R_1$ that is quite easy -- $F \rightarrow C$ -- and that's in BCNF. Our remaining FDs for $R_2$ are $\left\{BH \rightarrow AD, D \rightarrow BH, BE \rightarrow F, A \rightarrow GEF \right\}$.
        \item Decompose using $A \rightarrow GEF$. $A^+ = EFG$, so we have $R_3 = AEFG$ and $R_4 = ABDH$. The FD projection on $R_3$ is $A \rightarrow EFG$, and that's in BCNF. Our remaining FDs for $R_4$ are $\left\{BH \rightarrow AD, D \rightarrow BH \right\}$.
        \item $R_4$ is in BCNF, since both the FDs include all other columns.
        \item At the end, we have:
            \begin{equation*}
            \begin{aligned}
            R_1 =& FC, \left\{F \rightarrow C \right\} \\
            R_3 =& AEFG, \left\{A \rightarrow EFG \right\} \\
            R_4 =& ABDH, \left\{BH \rightarrow AD, D \rightarrow BH \right\} \\
            \end{aligned}
            \end{equation*}
        \end{itemize}
    \end{enumerate}
\item
    \begin{enumerate}
    \item [(a)] Using the closure test, we find that:
        \begin{itemize}
        \item $DBE^+$ includes all columns, $DBE$ is a superkey.
        \item $CD^+$ includes all columns, $CD$ is a superkey.
        \item $D^+$ includes all columns, $D$ is a superkey.
        \item $BADE^+$ includes all columns, $BADE$ is a superkey.
        \item $ABD^+$ includes all columns, $ABDBADE$ is a superkey.
        \item $EF^+$ does not include all columns, it is not a key.
        \end{itemize}
        Further, we can see that $D$ is a candidate key.
    \item [(b)] We were given:
            \begin{equation*}
            \begin{aligned}
            & \big\{DBE \rightarrow FC, CD \rightarrow AF, D \rightarrow AB, D \rightarrow G, \\
            & \qquad BADE \rightarrow C, ABD \rightarrow E, D \rightarrow F, EF \rightarrow B \big\} \\
            \end{aligned}
            \end{equation*}
        Via expansion of multivalue RHSes.
            \begin{equation*}
            \begin{aligned}
            & \big\{DBE \rightarrow C, DBE \rightarrow F, CD \rightarrow A, CD \rightarrow F, \\
            & \qquad D \rightarrow A, D \rightarrow B, D \rightarrow G, BADE \rightarrow C, \\
            & \qquad ABD \rightarrow E, D \rightarrow F, EF \rightarrow B \big\}
            \end{aligned}
            \end{equation*}
        It's now apparent that D's 1-to-1 relationships make many FDs redundant. Let's prune them.
            \begin{equation*}
            \begin{aligned}
            & \text{First, remove directly redundant things:} \\
            & \big\{DBE \rightarrow C, D \rightarrow A, D \rightarrow B, D \rightarrow G, \\
            & \qquad BADE \rightarrow C, ABD \rightarrow E, D \rightarrow F, EF \rightarrow B \big\} \\
            & \text{From D we have A and B, so from D we have E.} \\
            & \big\{DBE \rightarrow C, D \rightarrow A, D \rightarrow B, D \rightarrow G, \\
            & \qquad BADE \rightarrow C, D \rightarrow E, D \rightarrow F, EF \rightarrow B \big\} \\
            & \text{Then, we also have C...} \\
            & \big\{D \rightarrow A, D \rightarrow B, D \rightarrow C, D \rightarrow E, \\
            & \qquad D \rightarrow F, D \rightarrow G, EF \rightarrow B \big\}
            \end{aligned}
            \end{equation*}
        Thus, we have found the minimal basis.
    \item [(c)] From the minimal basis, we end with two relations:
        \begin{itemize}
        \item $R_1$ with columns $EFB$ and FDs $EF \rightarrow B$
        \item $R_2$ with columns $ACDEF$ and FDs $D \rightarrow ACEF$.
        \end{itemize}
    \item [(d)] The schema does not allow redundancy.
    \end{enumerate}
\end{enumerate}

\end{document}

\documentclass{article}
\usepackage{amssymb,graphicx}
\usepackage{hyperref}
\usepackage{tikz}
\usepackage[fleqn]{amsmath}
\usetikzlibrary{trees}
\title{CSC353 Fall 2015, Assignment 1}
\author{Connor Peet \#1001088208}
\renewcommand{\today}{~}
\hypersetup{pdfpagemode=Fullscreen,
  colorlinks=true,
  linkfileprefix={}}
\newcommand{\floor}[1]{\lfloor #1\rfloor}
\begin{document}
\maketitle

\section*{Part 1: Queries}

\begin{enumerate}

\item   % ----------
Find the last names of the athlete(s) of the country(ies) that did not compete in any event yet.

{\bf Answer}:\\[5pt]
\begin{equation}
\begin{aligned}
Answer := \Pi_{lname} [(\Pi_{EID} \mathrm{Athlete} - \Pi_{EID} \mathrm{Result}) \bowtie \mathrm{Athlete}]
\end{aligned}
\end{equation}

\item   % ----------
Find the last names of the athlete(s) of the country(ies) that did not win any medals yet (either because they did not compete, or because their athletes did not rank in the top 3 in any event so far).

{\bf Answer}:\\[5pt]
\begin{equation}
\begin{aligned}
Answer := \Pi_{lname} \sigma_{gold=silver=bronze=0} \mathrm{Athlete}
\end{aligned}
\end{equation}


\item   % ----------
Find the stadium names of all the stadiums where exactly one event took place.

{\bf Answer}:\\[5pt]
\begin{equation}
\begin{aligned}
Thrice & := \sigma_{E1.SID=E2.SID=E3.SID \land E1.EID \neq E2.EID \neq E3.EID}\\
    & \quad \quad \quad (\rho_{E1} \mathrm{Event} \times \rho_{E2} \mathrm{Event} \times \rho_{E3} \mathrm{Event})\\
Answer & := \Pi_{SID} \mathrm{Event} - \Pi_{SID} \mathrm{Thrice}
\end{aligned}
\end{equation}


\item   % ----------
Find all the sporting disciplines that Canadian athletes have competed in so far.

{\bf Answer}:\\[5pt]
\begin{equation}
\begin{aligned}
Answer := \Pi_{Event.sport} \sigma_{cname=Canada} (\mathrm{Athlete} \bowtie \mathrm{Country} \bowtie \mathrm{Result} \bowtie \mathrm{Event})
\end{aligned}
\end{equation}

\item   % ----------
Find the first and last name of the athletes whose sporting discipline is ``swimming''
and who have won the highest number of gold medals among all athletes who compete
in the same sport.

{\bf Answer}:\\[5pt]
\begin{equation}
\begin{aligned}
Swimmers := & \sigma_{sport=swimming} \mathrm{Athlete} \\
Lowers := & \sigma_{S1.gold < S2.gold} (\rho_{S1} \mathrm{Swimmers} \times \rho_{S2} \mathrm{Swimmers}) \\
Answer := & \Pi_{fname, lname} \mathrm{Swimmers} - \Pi_{fname, lname} \mathrm{Lowers}
\end{aligned}
\end{equation}

\item   % ----------
Find the name of every country that has won at least one of every type of medal
(gold, silver, and bronze).

{\bf Answer}:\\[5pt]
\begin{equation}
\begin{aligned}
Winners := & [\rho_{A1} (\sigma_{gold>0} \mathrm{Athlete}) \bowtie_{A1.CID=A2.CID} \\
    & \quad \quad \rho_{A2} (\sigma_{silver>0} \mathrm{Athlete}) \bowtie_{A2.CID=A3.CID} \\
    & \quad \quad \rho_{A3} (\sigma_{bronze>0} \mathrm{Athlete})] \\
Answer := & \Pi_{cname} (\mathrm{Winners} \bowtie \mathrm{Country})
\end{aligned}
\end{equation}


\item   % ----------
Find the gold medalist country of the event for which the very first ticket out of
all the tickets in the database was purchased. A gold medalist country is a country
that has won at least one gold medal.

{\bf Answer}:\\[5pt]
\begin{equation}
\begin{aligned}
First := & \mathrm{Ticket} - \Pi_{T2.*} \sigma_{T2.dateIssued > T1.dateIssued} \\
    & \quad \quad (\rho_{T1} \mathrm{Ticket} \times \rho_{T2} \mathrm{Ticket}) \\
Answer := & \Pi_{CID} \sigma_{medal=gold}(\mathrm{First} \bowtie \mathrm{Result} \bowtie \mathrm{Athlete})
\end{aligned}
\end{equation}


\item   % ----------
Find the first and last name of the athlete representing ``Mexico'', who so far
has the second highest number of gold medals (among athletes of the same country).

{\bf Answer}:\\[5pt]
\begin{equation}
\begin{aligned}
Mexican := & \sigma_{cname=Mexico} (\mathrm{Athlete} \bowtie \mathrm{Country}) \\
Bottom1 := & \Pi_{M2.*} \sigma_{M1.gold < M2.gold} (\rho_{M1} \mathrm{Mexican} \times \rho_{M2} \mathrm{Mexican}) \\
Bottom2 := & \Pi_{B2.*} \sigma_{B1.gold < B2.gold} (\rho_{B1} \mathrm{Bottom1} \times \rho_{B2} \mathrm{Bottom1}) \\
Second := & \mathrm{Bottom1} - \mathrm{Bottom2} \\
Answer := & \Pi_{fname, lname} Second
\end{aligned}
\end{equation}


\item   % ----------
Find the sports disciplines for events for which at least two tickets
were bought on the date of the event.

{\bf Answer}:\\[5pt]
\begin{equation}
\begin{aligned}
Late := & \sigma_{Ticket.date=Event.date} (\mathrm{Ticket} \bowtie \mathrm{Event}) \\
Doubles := & \Pi_{P1.EID} \sigma_{P1.TID \neq P2.TID \land P1.EID = P2.EID} (\rho_{P1} \mathrm{Late} \times \rho_{P2} \mathrm{Late}) \\
Answer := & \Pi_{sport} (\mathrm{Doubles} \bowtie \mathrm{Event})
\end{aligned}
\end{equation}

\item   % ----------
Find the athlete with the highest overall number of gold medals won so far, and report that athlete's first and last name, country name, and number of gold medals won.

{\bf Answer}:\\[5pt]
\begin{equation}
\begin{aligned}
Bottom := & \Pi_{A1.*} \sigma_{A1.gold < A2.gold} (\rho_{A1} \mathrm{Athlete} \times \rho_{A2} \mathrm{Athlete}) \\
Answer := & \Pi_{fname, lname, cname, gold} [(\mathrm{Athlete} - \mathrm{Bottom}) \bowtie \mathrm{Country}]
\end{aligned}
\end{equation}

\item   % ----------
Find the discipline (sport) of the event for which the highest number of tickets was purchased.

TODO: we've not learned aggregate functions yet

{\bf Answer}:\\[5pt]
\begin{equation}
\begin{aligned}
Lonely := & \Pi_{EID} Event - \Pi_{EID} Ticket\\
Answer := & \Pi_{fname, lname} \sigma_{medal=gold} (\mathrm{Lonely} \bowtie \mathrm{Result} \bowtie \mathrm{Athlete})
\end{aligned}
\end{equation}

\item   % ----------
Find the first and last name for all athletes who have won a gold medal in an event for which no tickets were sold.

{\bf Answer}:\\[5pt]
\begin{equation}
\begin{aligned}
Lonely := & \Pi_{EID} Event - \Pi_{EID} Ticket\\
Answer := & \Pi_{fname, lname} \sigma_{medal=gold} (\mathrm{Lonely} \bowtie \mathrm{Result} \bowtie \mathrm{Athlete})
\end{aligned}
\end{equation}

\end{enumerate}



%----------------------------------------------------------------------------------------------------------------------
\section*{Part 2: Additional Integrity Constraints [16\% - 4 marks each]}

Below are some additional integrity constraints on our schema. Express each of them
using the notation from Section 2.5 of your textbook. If a constraint cannot be
expressed using such notations, simply write ``cannot be expressed''.


\begin{enumerate}

\item   % ----------
An athlete cannot win more than one medal type in the same event.

{\bf Answer}:\\[5pt]
{
Example:\\[5pt]
$
R = \emptyset\\[5pt]
$
This is just an exampls, so please replace this \\[5pt]
with the correct answer.
}

\item   % ----------
All tickets for an event have to be purchased before the time of the event.

{\bf Answer}:\\[5pt]
{
$
Answer here
$
}


\item   % ----------
The number of tickets purchased for an event should not exceed the capacity
of the stadium where the event takes place.

{\bf Answer}:\\[5pt]
{
$
Answer here.
$
}

\item   % ----------
An athlete could not have competed in an event for a sporting discipline
that they are not qualified to participate in.

{\bf Answer}:\\[5pt]
{
$
Answer here.
$
}

\end{enumerate}


\end{document}



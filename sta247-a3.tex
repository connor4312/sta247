\documentclass{article}
\usepackage{amsmath,amssymb,graphicx,tikz}
\usepackage{hyperref}
\usepackage{tikz}
\usepackage{amsmath}
\usetikzlibrary{trees}
\title{STA247 Fall 2015, Assignment 3}
\author{Connor Peet \#1001088208}
\renewcommand{\today}{~}
\hypersetup{pdfpagemode=Fullscreen,
  colorlinks=true,
  linkfileprefix={}}
\newcommand{\floor}[1]{\lfloor #1\rfloor}
\begin{document}
\maketitle

\begin{enumerate}
\item The length of time for a student to be served at the university bookstore is a random variable having an exponential distribution with a mean of 4 minutes.
    \begin{itemize}
    \item [(a)] Find the probability that a student is served in less than 3 minutes.
        \begin{itemize}
        \item We know the pdf is defined as $f(x) = \lambda e^{-\lambda x}$, and $\mu = \frac{1}{\lambda}$.
        \item So the cdf is $F(x) = \int_0^x f(x) dx = 1 - e^{-\lambda x}$, and $\lambda = \frac{1}{4}$.
        \item So then $P(X < 3)$ is simply $F(3) = 1 - e^{-3/4} \approx 0.5276$
        \end{itemize}
    \item [(b)] Find the probability that a student is served in less than 3 minutes on at least 4 of the next 6 days?
        \begin{itemize}
        \item The serving times are independent, so we're essentially being asked for the chance that an event with a probability of 0.5276 will occur at least four times in six trials. In other words, $X \sim Bin(6, 1 - e^{-3/4})$.
        \item So $P(X >= 4)$ could be written in R as $\mathrm{pbinom}(4, size=6, prob=(1 - exp(-3/4)) \approx 0.8625206$
        \end{itemize}
    \end{itemize}

\item Let  $X_1,X_2, X_3$ be three independent variables from the exponential distribution with mean 4.
    \begin{itemize}
    \item [(a)] Write the pdf of $T=X_1+X_2+X_3$.
        \begin{itemize}
        \item We know $f(x) = \frac{1}{4} e^{-\frac{x}{4}}$ (where $\lambda = \frac{1}{4}$).
        \item Then $g(x) = 3f(x) = \frac{3}{4} e^{-\frac{3x}{4}}$.
        \end{itemize}
    \item [(b)] Find $P(T>7)$. Use R to find the answer.
        \begin{itemize}
        \item $P(T > 7) = 1 - \int_0^7 \frac{1}{4} e^{-\frac{3x}{4}} dx$
        \item Then $P(T > 7) = \frac{2}{3} + \frac{1}{3} e^{-21/4} \approx 0.6684$.
        \end{itemize}
    \item [(c)] Find $E(T/3)$ and $Var(T/3)$.
        \begin{itemize}
        \item $E(T/3) = \int_{-\infty}^{\infty} \frac{x}{3} g(x) dx$.
        \item Then $E(T/3) \approx$
        \end{itemize}
    \end{itemize}

\item Question 3:
    \begin{itemize}
    \item [(a)] Find $\Gamma\left(\frac{5}{2}\right)$.
        \begin{equation*}
        \begin{aligned}
        \Gamma\left(\frac{5}{2}\right) = & \left(\frac{3}{2}\right) \Gamma\left(\frac{3}{2}\right) \\
            = & \left(\frac{3}{2}\right) \left(\frac{1}{2}\right) \Gamma\left(\frac{1}{2}\right) \\
            = & \left(\frac{3}{2}\right) \left(\frac{1}{2}\right) \sqrt{\pi} \\
            = & \frac{3\sqrt{\pi}}{4} \approx 1.329
        \end{aligned}
        \end{equation*}
    \end{itemize}
    \item [(b)] Use part (a) and the pdf of the gamma distribution  to find the exact value of $\int_{0}^{\infty}x^{1.5}e^{-4x}dx$.
        \begin{itemize}
        \item Say that $y = \int_{0}^{\infty}x^{1.5}e^{-4x}dx$.
        \item Then $\frac{4^{2.5}}{\Gamma(2.5)} y = \int_{0}^{\infty} \frac{4^{2.5}}{\Gamma(2.5)} x^{1.5}e^{-4x} dx$
        \item Then it's clear the function $f(x)$ inside the integral describes $X \sim Gamma(2.5, 4)$, so $\int_{0}^{\infty} f(x) dx = 1$.
        \item Then $\frac{4^{2.5}}{\Gamma(2.5)} y = 1$
        \item Then $y = \int_{0}^{\infty}x^{1.5}e^{-4x}dx = \frac{\Gamma(2.5)}{4^{2.5}} \approx 0.0415$
        \end{itemize}

\item [4.] Use the pdf of the normal distribution to find the exact value of the following integrals.
    \begin{itemize}
    \item $\int_{-\infty}^{\infty}e^{-\frac{1}{8}(x+5)^2}dx$.
        \begin{itemize}
        \item We can manipulate this into what looks like the equation of a normal distribution.
        \begin{equation*}
            \begin{aligned}
            y = & \int_{-\infty}^{\infty}e^{-\frac{1}{8}(x+5)^2} \\
            \frac{y}{\sqrt{2 \pi} 2} = & \int_{-\infty}^{\infty} \frac{1}{\sqrt{2 \pi} 2} e^{-(x - (-5))^2 / (2 \times 2^2)}
            \end{aligned}
        \end{equation*}
        \item Then the right-hand side is the cumulative distribution of $Y \sim N(-5, 4)$, so it evaluates to $1$.
        \item Then $y = \int_{-\infty}^{\infty}e^{-\frac{1}{8}(x+5)^2}dx = 2 \sqrt{2 \pi} \approx 5.0133 $
        \end{itemize}
    \item $\int_{-\infty}^{\infty}(x^2-x+3)e^{-\frac{1}{8}(x+5)^2}dx$.
        \begin{itemize}
        \item First, we can split this into multiple integrals. Then we can recognize the expected value from it, and use the values found in part (a) to solve it.
        \begin{equation*}
            \begin{aligned}
            w = & \int_{-\infty}^{\infty}(x^2-x+3)e^{-\frac{1}{8}(x+5)^2}dx \\
            w = & \int_{-\infty}^{\infty}x^2 e^{-\frac{1}{8}(x+5)^2} - x e^{-\frac{1}{8}(x+5)^2} + 3 e^{-\frac{1}{8}(x+5)^2}dx \\
            w = & 2\sqrt{2 \pi} \int_{-\infty}^{\infty}x^2 \frac{1}{2\sqrt{2 \pi}} e^{-\frac{1}{8}(x+5)^2} \\
                & \quad - 2\sqrt{2 \pi} \int_{-\infty}^{\infty} x \frac{1}{2\sqrt{2 \pi}} e^{-\frac{1}{8}(x+5)^2} \\
                & \quad + 3\int_{-\infty}^{\infty} e^{-\frac{1}{8}(x+5)^2}dx \\
            w = & 2\sqrt{2 \pi}(E(Y^2) - E(Y)) + 3y \\
            w = & 2\sqrt{2 \pi}((4 + 5^2) + 5 + 3) \\
            w = & 74 \sqrt{2 \pi} \approx 185.5
            \end{aligned}
        \end{equation*}
        \end{itemize}
    \item $\int_{-\infty}^{\infty}e^{-\frac{1}{8}x^2+6x+2}dx$.
        \begin{itemize}
        \item We complete the square, then it all flows down nicely like part (a).
        \begin{equation*}
            \begin{aligned}
            z = & \int_{-\infty}^{\infty} e^{-\frac{1}{8}(x^2 - 48x - 16)}dx \\
            z = & \int_{-\infty}^{\infty} e^{-\frac{1}{8}(x^2 - 48x +576) + \frac{592}{8}}dx \\
            z = & \int_{-\infty}^{\infty} e^{\frac{592}{8}} e^{-\frac{1}{8}(x - 24)^2}dx \\
            \frac{z}{2\sqrt{2 \pi} e^{\frac{592}{8}}} = & \int_{-\infty}^{\infty} \frac{1}{\sqrt{2 \pi} 2} e^{-\frac{1}{8}(x - 24)^2}dx = 1 \\
            z = & 2\sqrt{2 \pi} e^{\frac{592}{8}} \approx 6.885 \times 10^{32}
            \end{aligned}
        \end{equation*}
        \end{itemize}
    \end{itemize}

\item The grades of a certain probability exam are normally distributed with mean 75 and standard deviation 5.
    \begin{itemize}
    \item [(a)] Find the minimum grade for the top 10\% of the grades.
        \begin{itemize}
        \item We look on the normal distribution chart and find that the $Z$ in $P(0 \leq X \leq Z) = 0.4$ (and therefore $P(X > Z) = 0.10$) is $Z \approx 1.28$.
        \item Then we plug that into the equation $Z = \frac{X - \mu}{\sigma}$ to find that the necessary grade $X = 81.4$.
        \end{itemize}
    \item [(b)] Find the proportion of grades that are greater than 80.
        \begin{itemize}
        \item Straightforward enough. $P(X > 80) = P(Z > \frac{80 - 75}{5}) = 0.5 - P(0 \leq Z \leq \frac{80 - 75}{5})$
        \item Then $P(X > 80) = 0.5 - 0.3413 = 0.1587$.
        \end{itemize}
    \item [(c)] Ten grades are randomly selected, find the probability that at least two of these ten grades are each greater than 80.
        \begin{itemize}
        \item The distribution is modeled as $X \sim Binomial(10, 0.1587)$. Then $P(X \geq 2) = 1 - P(X = 0) - P(X = 1) = 0.4873$ (found using R).
        \end{itemize}
    \item [(d)] Hundred grades are randomly selected, find the probability that at least 20 of these grades are each greater than 80.
        \begin{itemize}
        \item This can be approximated using a normal distribution. We know $X \sim Binomial(100, 0.1587)$, so the approximation is $X \sim Normal(\mu = 15.87, \sigma^2 = 13.35)$.
        \item Then we want $P(Z > \frac{20 - 15.87}{\sqrt{13.35}}) = 0.5 - P(0 \leq Z < \frac{20 - 15.87}{\sqrt{13.35}}) = 0.5 - 0.3708 = 0.1292$
        \end{itemize}
    \item [(e)] Use R to find the exact value in part (d).
        \begin{itemize}
        \item $1 - pbinom(20, size=100, prob=0.1587) = 0.1052639$
        \end{itemize}
    \end{itemize}
\end{enumerate}

\end{document}

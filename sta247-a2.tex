\documentclass{article}
\usepackage{amssymb,graphicx}
\usepackage{hyperref}
\usepackage{pgfplots}
\usepackage[fleqn]{amsmath}

\pgfplotsset{compat=1.6}
\pgfplotsset{soldot/.style={only marks,mark=*}}
\pgfplotsset{holdot/.style={fill=white,only marks,mark=*}}

\title{STA247 Fall 2015, Assignment 2}
\author{Connor Peet \#1001088208}
\renewcommand{\today}{~}
\hypersetup{pdfpagemode=Fullscreen,
  colorlinks=true,
  linkfileprefix={}}
\newcommand{\floor}[1]{\lfloor #1\rfloor}
\begin{document}
\maketitle

\begin{enumerate}
\item For the probability distribution
    \begin{table}[h] \centering
      \begin{tabular}{c|c|c|c|c}
        $x$ & 0 & 1 & 2 & 3 \\ \hline
        $p(x)$& 0.3 & 0.4 & 0.2&$k^2$
      \end{tabular}
    \end{table}
    \begin{enumerate}
    \item [(a)] Find: $k$
        \begin{itemize}
        \item We know that the total probability must equal 1, so therefore $0.3 + 0.4 + 0.2 + k^2 = 1$.
        \item Then $k = \sqrt{1 - 0.2 - 0.4 - 0.3} \approx 0.3162$
        \end{itemize}
    \item [(b)] Find $P(0< X \le 2)$:
        \begin{itemize}
        \item{itemize} To do this, we can simply add the discrete probabilities $p(0) + p(1) + p(2)$.
        \item{itemize} Then $P(0< X \le 2) = 0.3 + 0.4 + 0.2 = 0.9$
        \end{itemize}
    \item [(c)] Find $F(x)$, the cumulative distribution function:
        \begin{equation*}
        F(x) = \begin{cases}
            0.3, & x < 1 \\
            0.7, & 1 \leq x < 2 \\
            0.9, & 2 \leq x < 3 \\
            1,   & 3 \leq x < 4 \\
        \end{cases}
        \end{equation*}
    \item [(d)] Graph $F(x)$: \\
        \begin{tikzpicture}
            \begin{scope}
                \begin{axis}[xlabel={x},ylabel=F(x)]
                    \addplot[domain=0:1]{0.3};
                    \addplot[domain=1:2]{0.7};
                    \addplot[domain=2:3]{0.9};
                    \addplot[domain=3:4]{1};
                    \addplot[holdot] coordinates{(1,0.3)(2,0.7)(3,0.9)(4,1)};
                    \addplot[soldot] coordinates{(0,0.3)(1,0.7)(2,0.9)(3,1)};
                \end{axis}
            \end{scope}
        \end{tikzpicture}
        \end{enumerate}
    \item [(e)] Find $E(X)$:
        \begin{itemize}
        \item The expected value is simply the weighted average of all values.
        \item This can be found with $E(X) = (0)(0.3) + (1)(0.4) + (2)(0.2) + (3)(0.1) = 1.1$.
        \end{itemize}
    \item [(d)] Find $Var(X)$;
        \begin{itemize}
        \item The variance can be expressed as $Var(X) = E(X^2) - E(X)$.
        \item Therefore the variance is $Var(X) = E(X^2) - E(X)= (0.3)(0)^2 + (0.4)(1)^2 + (0.2)(2)^2 + (0.1)(3)^2 - 1.1= 1$.
        \end{itemize}

\item Let X be a discrete random variable with the following a cumulative distribution function.
    \begin{itemize}
    \item [(a)] $P(X = 5)$
        \begin{itemize}
        \item The answer is contained in the given function: $\frac{3}{4} - \frac{1}{2} = \frac{1}{4}$
        \end{itemize}
    \item [(b)] $P(1.3 < X < 6)$:
        \begin{itemize}
        \item This is a discrete distribution, so then $P(1.3 < X < 3) = 0$ and $P(5 \leq X < 6) = P(5 \leq X < 7)$.
        \item Then $P(1.3 < x < 6) = P(1.3 < X < 3) + P(3 \leq X < 5) + P(5 \leq X < 7) = 0 + \frac{1}{4} + \frac{1}{4} = \frac{1}{2}$
        \end{itemize}
    \item [(c)] $P(X \le 5|X \ge 2)$:
        \begin{itemize}
        \item Again, since this is discrete, $P(X \geq 2) = P(X \geq 3)$.
        \item So $P(X \le 5|X \ge 2)$ = $\frac{P(X = 5) + P(X = 3)}{P(X = 3) + P(X = 5) + P(X = 7)} = \frac{1/2}{3/4} = \frac{3}{8}$.
        \end{itemize}
    \item [(d)] $E(X) = (0)(0) + (\frac{1}{4})(1) + (\frac{1}{4})(3) + (\frac{1}{4})(5) + (\frac{1}{4})(7) = 4$
    \item [(e)] $Var(x) = E(X^2) - E(X) = (0)(0) + (\frac{1}{4})(1)^2 + (\frac{1}{4})(3)^2 + (\frac{1}{4})(5)^2 + (\frac{1}{4})(7)^2  - 4 = 21 - 4 = 17$
    \end{itemize}

\item Let $X$ be the number of times a basketball player scores in free throws. Suppose that the probability that he scores at least once in six free throws is equal to 0.999936.
    \begin{enumerate}
    \item [(a)] Find the probability mass function of  $X$.
        \begin{itemize}
        \item The probability of scoring at least one in six throws is $P(x \geq 1) = 1 - P(0) = 0.999936$.
        \item Then $P(0) = 0.000064 = (1 - p)^{6}$.
        \item Then $p = \frac{4}{5}$.
        \item So the probability mass function is $p(x) = \binom{n}{x} \frac{4}{5}^x \frac{1}{5}^{n-x}, x \in \mathbb{N}$
        \end{itemize}
    \item [(b)] Find $P(X \ge 3)$. [You can use R to find the answer]
        \begin{itemize}
        \item Using R, $pbinom(2, size=6, prob=0.2) = 0.90112$
        \end{itemize}
    \item [(c)] Find $E(X(X-1))$.
        \begin{itemize}
        \item Note that $E(X(X - 1)) = E(X^2 - X) = E(X^2) - E(X)$.
        \item Then $E(X^2) - E(X) = np(1 - p) + n^2p^2 - np$ using definitions from the lectures.
        \item Then $E(X^2) - E(X) = np(1 - p + np - 1) = np^2(n - 1)$
        \item Then $E(X^2) - E(X) = n(\frac{4}{5})^2(n - 1)$
        \item Then $E(X(X - 1)) = n\frac{16}{25}(n - 1)$
        \end{itemize}
    \end{enumerate}

\item Suppose $X\sim Poisson (\lambda)$.
    \begin{itemize}
    \item [(a)] Show that $p(k+1)=\frac{\lambda}{k+1}p(k)$, where $p(k)=P(X=k)$.
        \begin{itemize}
        \item This can be proven like any recursive algorithim. For $k \in \mathbb{N}$, define $D(k)$ to be that $p(k)=P(X=k)$.
        \item \underline{Base case}: $D(0)$.
            \begin{itemize}
            \item Note that $0, 1 \in \mathbb{N}$.
            \item Then $p(0) = \frac{\lambda^0 e^{-\lambda}}{0!} = e^{-\lambda}$ by definition of a Poisson distribution.
            \item Also $p(1) = \frac{\lambda^1 e^{-\lambda}}{1!} = \frac{\lambda}{0 + 1} e^{-\lambda}$ by the same definition.
            \item Then $p(1) = p(0 + 1) = \frac{\lambda}{0+1}p(0)$.
            \item Then for $k = 0, D(0)$.
            \end{itemize}
        \item \underline{Inductive step}: Assume $k \in \mathbb{N}, D(k)$
            \begin{itemize}
            \item Then $p(k) = \frac{\lambda^k e^{-\lambda}}{k!}$ by IH and definition of a Poisson distribution.
            \item Then $p(k+1) = \frac{\lambda^{k+1} e^{-\lambda}}{(k+1)!}$.
            \item Then $p(k+1) = \frac{\lambda \lambda^k e^{-\lambda}}{(k+1)(k)!}$.
            \item Then $p(k+1) = \frac{\lambda}{k+1} \frac{\lambda^k e^{-\lambda}}{k!}$.
            \item Then $p(k+1) = \frac{\lambda}{k+1} p(k)$.
            \item Then $D(k+1)$.
            \end{itemize}
        \item Then $p(k+1)=\frac{\lambda}{k+1}p(k)$.
        \end{itemize}
    \item [(b)] If $\lambda=2$, compute $p(0)$.
        \begin{itemize}
        \item In the base case of the above proof, we found $p(0) = e^{-\lambda}$.
        \item So for $\lambda=2$, $p(0) = e^{-2} \approx 0.1353$
        \end{itemize}
    \item [(c)] Use the recursive relation in (a) and and $p(0)$ in (b), to find $p(1),p(2),p(3),$ and $p(4)$.
        \begin{itemize}
        \item $p(1) = \frac{2}{1+1}p(0) = e^{-2} \approx 0.1353$
        \item $p(2) = \frac{2}{3}p(1) = \frac{2}{3} e^{-2} \approx 0.0902$
        \item $p(3) = \frac{2}{4}p(2) = \frac{2}{4} \frac{2}{3} e^{-2} \approx 0.0451$
        \item $p(4) = \frac{2}{5}p(2) = \frac{2}{5} \frac{2}{4} \frac{2}{3} e^{-2} \approx 0.0180$
        \end{itemize}
    \end{itemize}
\item Questions:
    \begin{itemize}
    \item [(a)] A random variable $X$ has a binomial distribution, and a random variable $Y$ has a Poisson distribution. If $E(X)=E(Y)$, determine which random variable has the larger variance. Explain.
        \begin{itemize}
        \item If $E(X) = E(Y)$, then $np = \lambda$. Also, $V(Y) = \lambda$.
        \item Also $np(1 - p) < np$ since it simplifies to $-p < 0$, and $p \in \mathbb{R}, 0 \leq p \leq 1$.
        \item Then $np(1 - p) < \lambda$.
        \item Then $V(X) < V(Y)$.
        \end{itemize}
    \end{itemize}
\item Let $X \sim Gemoetric(p)$.
    \begin{enumerate}
    \item [(a)] Show that $P(X>k)=(1-p)^k$ for any integer $k$.
        \begin{itemize}
        \item We know that the probability distribution is $p(x) = p(1 - p)^{x-1}$.
        \item Therefore $p(x + 1) = p(1 - p)^x$, and the probability $P(X > k) = \sigma_{x=0}^k = \frac{r - r^{k+1}}{1-r}$

        \end{itemize}
    \item [(b)] Show that  $P(X> j+k|X > j)=P(X > k)$ for any integer $j,k\ge 0.$
    \end{enumerate}

\end{enumerate}

\end{document}
